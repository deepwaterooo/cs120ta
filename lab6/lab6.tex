% Created 2014-10-09 Thu 12:53
\documentclass[presentation]{beamer}
\usepackage[utf8]{inputenc}
\usepackage[T1]{fontenc}
\usepackage{fixltx2e}
\usepackage{graphicx}
\usepackage{longtable}
\usepackage{float}
\usepackage{wrapfig}
\usepackage{rotating}
\usepackage[normalem]{ulem}
\usepackage{amsmath}
\usepackage{textcomp}
\usepackage{marvosym}
\usepackage{wasysym}
\usepackage{amssymb}
\usepackage{hyperref}
\tolerance=1000
\usepackage{CJK}
\usepackage{indentfirst}
\usepackage[pdftex]{graphicx}
\usepackage{fancyhdr}
\usepackage[CJKbookmarks=true]{hyperref}
%% Define a museincludegraphics command, which is
%%   able to handle escaped special characters in image filenames.
\def\museincludegraphics{%
  \begingroup
  \catcode`\|=0
  \catcode`\\=12
  \catcode`\#=12
  \includegraphics[width=0.75\textwidth]
}
\usetheme{default}
\author{Heyan Huang}
\date{\today}
\title{CS120 Lab \alert{6} Section \alert{6}}
\hypersetup{
  pdfkeywords={},
  pdfsubject={},
  pdfcreator={Emacs 24.3.1 (Org mode 8.2.7c)}}
\begin{document}

\maketitle

\begin{frame}[fragile,label=sec-1]{Quiz for Week 5 \alert{Answers}}
 Write the prototype/declaration and the code/definition for a function that takes a double as input and returns the square of the double as the function's output. The rest of the program, including main(), is given.
\begin{verbatim}
#include <iostream>
#include <cmath>
using namespace std;
double my_function(double);
int main(){
    double x = 4.5;
    double y;
    y = my function(x);
    cout << y << endl;
}
double my_function(double x) {
    // return pow(x, 2);
    return x * x;
}
\end{verbatim}
\end{frame}

\begin{frame}[fragile,label=sec-2]{Quiz for Week 6 \alert{Answers}}
 All of the following questions are based on the class position, which is defined as:
\begin{verbatim}
class position {
private:
    int x;
    int y;
public:
    int distance();
};
\end{verbatim}
\begin{itemize}
\item Write the code to declare a single object named \alert{center} of type position:

\alert{position center;}
\item Write code to have the object named center call the \alert{distance()} function.

\alert{center.distance();}
\end{itemize}
\end{frame}

\begin{frame}[fragile,label=sec-3]{Quiz for Week 6 \alert{Answers} (continued)}
 All of the following questions are based on the class position, which is defined as:
\begin{verbatim}
class position {
private:
    int x;
    int y;
public:
    int distance();
};
\end{verbatim}
\begin{itemize}
\item What are the data members of the class?

\alert{x and y.}
\item What are the member functions (methods) of the class?

\alert{distance().}
\item If there was a constructor function for this class, what would the name of the function be?

\alert{position()}
\end{itemize}
\end{frame}

\begin{frame}[label=sec-4]{Scores of Quiz Week 5}
\\
\begin{itemize}
\item \alert{Quiz for Week 5} Distribution:
\end{itemize}
\begin{center}
\begin{tabular}{lrrrrrrr}
\hline
Score & 0 & 1 & 2 & 3 & 4 & 5 & Missed\\
\hline
Section \alert{4} Count (22) & 4 & 4 & 1 & 2 & 7 & 3 & 1\\
\hline
Section \alert{6} Count (24) & 1 & 1 & 2 & 4 & 6 & 2 & 8\\
\hline
\end{tabular}
\end{center}
\end{frame}

\begin{frame}[label=sec-5]{Lab 6 Specific Requirements}
\begin{itemize}
\item \alert{cscheckin}:
\begin{itemize}
\item \alert{Source Program} only
\item program name: \alert{Lab6Sec6.cpp}
\end{itemize}
\end{itemize}
\\
\begin{itemize}
\item \alert{Hard Copy}:
\begin{itemize}
\item \alert{Source Program}: Lab6Sec6.cpp
\item \alert{Script Output} of the program: 
\begin{itemize}
\item make sure \alert{./a.out} execute the modified parts of your program.
\end{itemize}
\end{itemize}
\end{itemize}
\end{frame}
% Emacs 24.3.1 (Org mode 8.2.7c)
\end{document}