% Created 2014-09-25 Thu 10:35
\documentclass[presentation]{beamer}
\usepackage[utf8]{inputenc}
\usepackage[T1]{fontenc}
\usepackage{fixltx2e}
\usepackage{graphicx}
\usepackage{longtable}
\usepackage{float}
\usepackage{wrapfig}
\usepackage{rotating}
\usepackage[normalem]{ulem}
\usepackage{amsmath}
\usepackage{textcomp}
\usepackage{marvosym}
\usepackage{wasysym}
\usepackage{amssymb}
\usepackage{hyperref}
\tolerance=1000
\usepackage{CJK}
\usepackage{indentfirst}
\usepackage[pdftex]{graphicx}
\usepackage{fancyhdr}
\usepackage[CJKbookmarks=true]{hyperref}
%% Define a museincludegraphics command, which is
%%   able to handle escaped special characters in image filenames.
\def\museincludegraphics{%
  \begingroup
  \catcode`\|=0
  \catcode`\\=12
  \catcode`\#=12
  \includegraphics[width=0.75\textwidth]
}
\usetheme{default}
\author{Heyan Huang}
\date{\today}
\title{CS120 Lab \alert{4} Section \alert{6}}
\hypersetup{
  pdfkeywords={},
  pdfsubject={},
  pdfcreator={Emacs 24.3.1 (Org mode 8.2.7c)}}
\begin{document}

\maketitle

\begin{frame}[fragile,label=sec-1]{Quiz for Week 3 \alert{Answers}}
 \begin{itemize}
\item In C++ a library is a file containing code that can be used in other programs.

What't the difference between a \alert{library} and a \alert{header} ?

\url{http://stackoverflow.com/questions/924485/whats-the-difference-between-a-header-file-and-a-library}

\begin{verbatim}
#include<stdio.h>
main() {
    printf("haiii"); 
}
\end{verbatim}
\begin{itemize}
\item \alert{header} file: interface
\begin{itemize}
\item contains the function (printf) declaration;
\item during preprocessing, the printf function is replaced by the function declaration;
\end{itemize}
\item \alert{library} file: implementation
\begin{itemize}
\item contains the function (printf) definition;
\item during linking, the fucntion declaration is replaced with the function definition. Obviously, everything will be in object while linking.
\end{itemize}
\end{itemize}
\end{itemize}
\end{frame}

\begin{frame}[fragile,label=sec-2]{Quiz for Week 3 \alert{Answers} (continue)}
 \begin{itemize}
\item A conditional is used in a program to control whether or not a piece of code is executed.

To be exact, instead of "\alert{a piece of code}", it's "\alert{a block of code}".

\item The following code snippet will print It��s true. (Type Conversion)
\begin{verbatim}
int x = 15;
if(x + 20 > 2 * x){
    cout << "It's true";
}
\end{verbatim}
\end{itemize}
\end{frame}

\begin{frame}[fragile,label=sec-3]{Quiz for Week 3 \alert{Answers} (continue)}
 \begin{itemize}
\item The following code snippet will print the word here exactly ten times.
\begin{verbatim}
int x = 0;
while (x<10) {
    cout << "here";
}
\end{verbatim}
Missed the \alert{x += 1;} or \alert{x = x + 1;} statement part;

\item The following code snippet will print the value 2.
\begin{verbatim}
int x = 25;
double y = 10;
cout << x/y;
\end{verbatim}
\end{itemize}
\end{frame}

\begin{frame}[label=sec-4]{\alert{Quiz Week 3} Scores Distribution}
\begin{itemize}
\item \alert{Quiz for Week 3} Distribution
\end{itemize}
\begin{center}
\begin{tabular}{lrrrrr}
\hline
Score & 2 & 3 & 4 & 5 & Missed\\
\hline
Section \alert{4} Count (22) & 0 & 4 & \alert{12} & 5 & 1\\
\hline
Section \alert{6} Count (24) & \alert{2} & 4 & 4 & \alert{10} & 4\\
\hline
\end{tabular}
\end{center}

\begin{itemize}
\item \alert{Lab 2} \& \alert{Assignment 2 \& 3}: Haven't got access yet, zipped file was filtered out by UI mail system yesterday;
\item \alert{Lab 3}: Got grading criteria this morning;
\item \alert{Lab 3}, \alert{Quiz Week 4} will be handed back during coming lab for sure;
\item Will get these cscheckin files on hand and grade them as soon as possible.
\end{itemize}
\end{frame}

\begin{frame}[label=sec-5]{Lab 4 Specific Requirements}
\begin{itemize}
\item \alert{cscheckin}:
\begin{itemize}
\item \alert{Source Programs} only, with \alert{file names} specifically named to be: 
\begin{itemize}
\item \alert{piLab4Section4.cpp}
\item \alert{eLab4Section4.cpp}
\end{itemize}
\end{itemize}
\item \alert{Hard Copy}:
\begin{itemize}
\item \alert{Source Programs}
\item \alert{Outputs} of the programs
\end{itemize}
\end{itemize}
\end{frame}
% Emacs 24.3.1 (Org mode 8.2.7c)
\end{document}