\def\spacingset#1{\renewcommand{\baselinestretch}{#1}\small\normalsize}
\documentclass[12pt,openbib]{article}
\usepackage{palatino}
\usepackage{epsfig}
\setlength{\textheight}{9.3in} 
\setlength{\textwidth}{6.50in} 
\setlength{\hoffset}{0.60in}
\hoffset= -.6875in 
%\voffset= -1.0in 
\setlength {\parskip}{9 pt} 
\setlength {\parindent}{0.0in}
\rm
\normalsize
\begin{document}
\thispagestyle{empty}
\vspace{-1.0in}
\large
\begin{center}
\textsf{CS120  Assignment 2}\\
%\textsf{Spring 2005}
\textsf{Fall 2014}
\end{center}
\vspace{-0.2in}
\large\textsf{Name: }\rule[-0.01in]{2.5in}{0.015in}
\hspace{0.5in} Section \rule[-0.01in]{1.0in}{0.015in}

\normalsize
\rm
%Time: 50  Minutes. Closed Book, There are 4 pages and 8 questions to this test. 
\vspace{0.5in}
\rule[-0.01in]{0.5in}{0.015in} User can enter first and last name (3 points)
 
\rule[-0.01in]{0.5in}{0.015in} Program refers to the user by name (2 points)

\rule[-0.01in]{0.5in}{0.015in} Uses the length of the user's name in calculating the lucky number (2 points)

\rule[-0.01in]{0.5in}{0.015in} Prints a different fortune for every two lucky numbers (3 points)

\rule[-0.01in]{0.5in}{0.015in} Block comment at the beginning of the program with the student's name, section number, assignment number, and data (no points for now)

\rule[-0.01in]{0.5in}{0.015in} Correct indenting and formatting to make the program easy to read (no points for now)

\rule[-0.01in]{0.5in}{0.015in} Clear variable names (no points for now)

\rule[-0.01in]{0.5in}{0.015in} Extra credit (up to 3 more points)

\rule[-0.01in]{0.5in}{0.015in} Total (out of 10)
\end{document}

