\def\spacingset#1{\renewcommand{\baselinestretch}{#1}\small\normalsize}
\documentclass[11pt]{article}
\usepackage{palatino}
%\usepackage{draftcopy}        % Puts the word DRAFT across the page
\usepackage[outline]{draftcopy}
\draftcopySetGrey{0.75}
\draftcopyName{Assign \#9 }{150}
\renewcommand{\ttdefault}{cmtt} % Sets the tt font back to the default
\font\fnta=cmr10 at 10pt       % Defines fonts for second part of handout
\font\fntb=cmti10 at 10pt      % Error message part uses "regular" fonts
\setlength{\textheight}{9.7in}
\setlength{\textwidth}{6.5in}
\setlength{\hoffset}{0.6in}
\hoffset= -0.675in
\voffset= -0.5in
\setlength {\parskip}{9 pt} 
\setlength {\parindent}{0.0in}
\rm
\normalsize
\pagestyle{empty}
\begin{document}

\begin{center}
\huge
\bf
{\sffamily CS120 - Computer Science I               \\
Assignment \#9~~~~~~ Fall   2014} 

\end{center}
\normalsize
\textbf{Due: Monday November 10th}
\\
\\
The purpose of this assignment is to give you some experience using two
dimensional arrays. You will also use a library to aid in  the display of
your output.

In 1970, British mathematician John Conway proposed the Game of Life,
a ``cellular automata'' game.
Each position in a 2-dimensional grid represents a living ``cell,'' which 
lives or dies in the next generation according to some simple rules involving
each cell's eight neighbors (N, NE, E, SE, S, SW, W, NW):

\small
\begin{enumerate}
 \item A living cell with fewer than two neighbors dies due to loneliness.
 \item A living cell with more than three neighbors dies due to overpopulation.
 \item A living cell with two or three neighbors lives to the next generation
 \item A cell is born if surrounded by exactly three living neighbors.
\end{enumerate}
\normalsize 

For this program, you are going to simulate Conway's Game of Life. Using a
2-D array of char, you will compute the next generation of the game, then 
display it. A living cell can be represented by a '1' in the position, and
a space otherwise. You should perform the calculation for the next generation
in a separate array, and once finished with the entire calculation, display
it.

The result of each new generation is best displayed on the screeen when it
does not scroll after each line. Therefore we will use a class called 
{\texttt CursWin}. This class provides capabilities very similar to the
standard iostream cout class. It includes numerous functions (methods) - the
complete documentation for the class is provided in the class files 
themselves. The main addition is that the user can ``move'' to a specific
place on the screen before outputting values. This allows the user to place
text anywhere on the screen. In our case, this will allow us to ``draw''
the game of life in place, rather than having it scroll off the screen as
we output.

The main features that you will need are as follows: 

\small
\begin{enumerate}
  \item You will need to {\texttt \#include "CursWin.h"} in your main program.
  \item  To make an output window, declare something like:
    \begin{verbatim}
     CursWin outw(0,0,20,70);
    \end{verbatim}
    This declares a Curses window with upper left (0,0) and lower right 
    (20,70) [(row,col)]
  \item Then write to the window, use it like you would a normal I/O stream, 
     ie:
    \begin{verbatim}
     outw << "Hello, world!" << cendl;
    \end{verbatim}
      The "normal" stream commands are included, prepended with a 'c', such 
      as "cendl" above.
   \item To move to a specific location within the window, use the Cmove(r,c) 
       macro:
    \begin{verbatim}
     outw << Cmove(10,20) << "This is the center of the screen" << cflush;
    \end{verbatim} 
   \item Something that is more important with CursWin than with iostream is 
         the use of cflush (which will output the text without
         appending a new line) or cendl. If you do not use cendl or cflush, 
         your 
         output will NOT get to the screen until the program encounters one 
         of these.  
   \item When you compile your program, you need to include both the Curswin.cpp
         file as well as the standard curses library:
    \begin{verbatim}
     g++ assign9.cpp CursWin.cpp -lcurses
    \end{verbatim}

\item The program should be able to read an initial configuration from a file.  This files contains the initial arrangement of alive and dead cells as a list of 900 (30x30 ``world'') of 1's and 0's.  When the program starts it should ask the user
if they want a random configuration or a configuration from a file.  If they pick random random configuration, the program should generate 1's and 0's (alive and dead cells) randomly.  If the user picks to load an initial configuration from a file, the program should ask for a file name, and the program should open that file and read in the initial configuration.  Two sample files are given on the course schedule.
\end{enumerate}

In addition to the proceeding required work, the following additions can be done for extra credit.  You may do as many or as few of these as you want.  Please include comments at the beginning of the program listing which of the extra credit problems you did.
\begin{enumerate}
\item Have the program also read the size of the ``world'' from the configuration.  I.e. the file should begin with two integers defining the width and height of the ``world''.   The program should read these first and create arrays, loops, etc. accordingly.

\item Have the program stop if it enters a constant configuration, i.e. if the configuration doesn't change between iterations.  This will require comparing two successive ``world'' states to see if they are indentical (i.e. have exactly the same 0's and 1's).  If two successive states are identical it means that the program has entered a steady-state and won't change any more.

\item Have the program  stop if it enters a steady-state with a period of two, i.e. if the configuration starts to repeat between on every other iteration.  This will require comparing ``world'' states that are one state apart to see if they are identical (i.e. have exactly the same 0's and 1's).  

\item Go on-line and find an alternative set of rules, either ones using only two states (alive and dead) or ones using three or more states.  Change the program to use the alternative rules and test it.   In the comment at the beginning of the program list the new rules and describe how the program behaved with them.
\end{enumerate}

\end{document}

