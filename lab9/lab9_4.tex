% Created 2014-10-28 Tue 12:21
\documentclass[presentation]{beamer}
\usepackage[utf8]{inputenc}
\usepackage[T1]{fontenc}
\usepackage{fixltx2e}
\usepackage{graphicx}
\usepackage{longtable}
\usepackage{float}
\usepackage{wrapfig}
\usepackage{rotating}
\usepackage[normalem]{ulem}
\usepackage{amsmath}
\usepackage{textcomp}
\usepackage{marvosym}
\usepackage{wasysym}
\usepackage{amssymb}
\usepackage{hyperref}
\tolerance=1000
\usepackage{CJK}
\usepackage{indentfirst}
\usepackage[pdftex]{graphicx}
\usepackage{fancyhdr}
\usepackage[CJKbookmarks=true]{hyperref}
%% Define a museincludegraphics command, which is
%%   able to handle escaped special characters in image filenames.
\def\museincludegraphics{%
  \begingroup
  \catcode`\|=0
  \catcode`\\=12
  \catcode`\#=12
  \includegraphics[width=0.75\textwidth]
}
\usetheme{default}
\author{Heyan Huang}
\date{\today}
\title{CS120 Lab \alert{9} Section \alert{4}}
\hypersetup{
  pdfkeywords={},
  pdfsubject={},
  pdfcreator={Emacs 24.3.1 (Org mode 8.2.7c)}}
\begin{document}

\maketitle

\begin{frame}[fragile,label=sec-1]{Quiz for Week 8 \alert{Answers}}
 (5 pt) Answer the following questions based on the following class declaration:
\begin{verbatim}
class circle {
  public:
    void print();
  private:
    int x;
    int y;
    float radius;
};
\end{verbatim}

\begin{itemize}
\item Write the code for a default constructor of the class circle that sets the location (x,y) to 0,0 and the radius to 1.
\begin{verbatim}
circle::circle() {
    x = 0;
    y = 0;
    radius = 1.0;
}
\end{verbatim}
\end{itemize}
\end{frame}

\begin{frame}[fragile,label=sec-2]{Quiz for Week 8 \alert{Answers} (continued)}
 \begin{verbatim}
class circle {
  public:
    void print();
  private:
    int x;
    int y;
    float radius;
};
\end{verbatim}
\begin{itemize}
\item Write the code for the print() function so that circles are printed as: 

\alert{A circle at x,y has radius r} 

Where x, y, and r are the values for the circle object.

\begin{verbatim}
void circle::print() {
    cout << "A circle at " << x << ", " << y
         << " has radius " << radius << endl;
}
\end{verbatim}
\end{itemize}
\end{frame}

\begin{frame}[fragile,label=sec-3]{Quiz for Week 8 \alert{Answers} (continued)}
 \begin{verbatim}
class circle {
  public:
    void print();
  private:
    int x;
    int y;
    float radius;
};
\end{verbatim}
\begin{itemize}
\item Write the statement(s) to create an array of 100 circles.

\alert{circle whatEverArrayNameHere[ 100 ];}
\item Write the statement(s) to print the 3rd object in the array of question 3.

\alert{whatEverArrayNameHere[ 2 ].print();}
\item Write the statement(s) to print the last object in the array of question 3.

\alert{whatEverArrayNameHere[ 99 ].print();}
\end{itemize}
\end{frame}

\begin{frame}[label=sec-4]{Scores of Quiz Week 8}
\\
\begin{itemize}
\item \alert{Quiz for Week 8} Distribution:
\end{itemize}
\begin{center}
\begin{tabular}{lrrrrrrr}
\hline
Score & 0 & 1 & 2 & 3 & 4 & 5 & Missed\\
\hline
Section \alert{4} Count (22) & 0 & 2 & 4 & 5 & 4 & 3 & 4\\
\hline
Section \alert{6} Count (24) & 0 & 3 & 5 & 4 & 5 & 0 & 7\\
\hline
\end{tabular}
\end{center}
\\
\begin{itemize}
\item \alert{Lab 7}:
\end{itemize}
\begin{center}
\begin{tabular}{lrrrrrrr}
\hline
Score & <9 & 10/.5 & 11/.5 & 12 & 12.5 & 13 & Missed\\
\hline
Section \alert{4} Count (22) & 1 & 5 & 1 &  & 6 & 3 & 6\\
\hline
Section \alert{6} Count (24) & 2 & 1 & 2 & 1 & 1 & 6 & 11\\
\hline
\end{tabular}
\end{center}
\\
\begin{itemize}
\item \alert{Lab 8}:
\begin{itemize}
\item Just got the grading criteria yesterday afternoon after after 4:00pm
\item Will hand it back to you during coming lab
\end{itemize}
\end{itemize}
\end{frame}
\begin{frame}[label=sec-5]{Emacs}
\begin{itemize}
\item The reason why I like Emacs:
\begin{itemize}
\item Emacs is an \alert{operating system};
\item Just like iphone/ipad changed human being's life, emacs \alert{makes programmer's life so much easier\textasciitilde{}!}
\item Most popular packages: 
\begin{itemize}
\item \alert{autocomplete}
\item \alert{yasnippet}
\item \alert{org-mode}
\item autopair
\item AUCTex
\end{itemize}
\end{itemize}
\item Personal Experience:
\begin{itemize}
\item when I took CS120 we don't have labs at all, and I have never been exposed to \alert{emacs} or \alert{vim} command based editors;
\item From my experience, I hope you guys can be exposed to these command-based editors \alert{as soon as possible}.
\end{itemize}
\item Contents that will be covered in this lab:
\begin{itemize}
\item \alert{basic commands} that emacs programmers use every day;
\item one line \alert{configuration} for line number
\end{itemize}
\end{itemize}
\end{frame}

\begin{frame}[label=sec-6]{Emacs Basic Commands}
\begin{itemize}
\item google "\alert{Emacs cheat sheet}"
\item Motion
\end{itemize}
\begin{center}
\begin{tabular}{lll}
\hline
entity to move over & backward & forward\\
\hline
character & C-b & C-f\\
word & M-b & M-f\\
line & C-p & C-n\\
go to line beginning/end & C-a & C-e\\
\hline
\end{tabular}
\end{center}
\begin{itemize}
\item other commands
\end{itemize}
\begin{center}
\begin{tabular}{ll}
\hline
search forward & C-s\\
search backward & C-r\\
scroll to next screen & C-v\\
scroll to previous screen & M-v\\
goto line & M-g g\\
set mark here & C-SPC or C-@\\
copy region to kill ring & M-w\\
yank back last thing killed & C-y\\
\hline
\end{tabular}
\end{center}
\end{frame}

\begin{frame}[label=sec-7]{Emacs configure \alert{Line Number}}
\begin{itemize}
\item command option
\begin{itemize}
\item works for current buffer
\item commands: \alert{M-x linum-mode RET}
\end{itemize}
\item permanent configuration in home directory
\begin{itemize}
\item go to home directory: \alert{pwd} ----> \alert{/home/huan3416}
\item \alert{man ls}, check \alert{-a}, \alert{-l}, \alert{-t} options; type \alert{q} to quit from \alert{man} results
\item \alert{ls -a}: check if there is an emacs configuration file named \alert{.emacs}
\item If not, create \alert{.emacs} file by typing \alert{emacs .emacs} in terminal
\item type \alert{(global-linum-mode 1)} in the file
\item \alert{C-x C-s} to \alert{save} the file
\item \alert{C-x C-c} to \alert{exit} from emacs
\end{itemize}
\item later on when you have questions, please try to use \alert{emacs}, or if you prefer to use \alert{vim}
\end{itemize}
\end{frame}

\begin{frame}[label=sec-8]{Lab 9 Specific Requirements}
\begin{itemize}
\item \alert{cscheckin}:
\begin{itemize}
\item \alert{Source Programs} only: \alert{Lab9Sec4.cpp}
\end{itemize}
\end{itemize}
\\
\begin{itemize}
\item \alert{Hard Copy}:
\begin{itemize}
\item \alert{Source Program}: 
\begin{itemize}
\item Lab9Sec4.cpp
\end{itemize}
\item \alert{Script Output} of the program
\end{itemize}
\end{itemize}
\end{frame}
% Emacs 24.3.1 (Org mode 8.2.7c)
\end{document}