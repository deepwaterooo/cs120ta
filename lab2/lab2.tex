% Created 2014-09-11 鍥� 10:43
\documentclass[11pt]{beamer}

      \mode<{{{beamermode}}}>

      \usetheme{{{{beamertheme}}}}

      \usecolortheme{{{{beamercolortheme}}}}

      \beamertemplateballitem

      \setbeameroption{show notes}
      \usepackage[utf8]{inputenc}

      \usepackage[T1]{fontenc}

      \usepackage{hyperref}

      \usepackage{color}
      \usepackage{listings}
      \lstset{numbers=none,language=[ISO]C++,tabsize=4,
  frame=single,
  basicstyle=\small,
  showspaces=false,showstringspaces=false,
  showtabs=false,
  keywordstyle=\color{blue}\bfseries,
  commentstyle=\color{red},
  }

      \usepackage{verbatim}

      \institute{{{{beamerinstitute}}}}
          
       \subject{{{{beamersubject}}}}

\usepackage[utf8]{inputenc}
\usepackage[T1]{fontenc}
\usepackage{fixltx2e}
\usepackage{graphicx}
\usepackage{longtable}
\usepackage{float}
\usepackage{wrapfig}
\usepackage{rotating}
\usepackage[normalem]{ulem}
\usepackage{amsmath}
\usepackage{textcomp}
\usepackage{marvosym}
\usepackage{wasysym}
\usepackage{amssymb}
\usepackage{hyperref}
\tolerance=1000
\usepackage{CJK}
\usepackage{indentfirst}
\usepackage[pdftex]{graphicx}
\usepackage{fancyhdr}
\usepackage[CJKbookmarks=true]{hyperref}
%% Define a museincludegraphics command, which is
%%   able to handle escaped special characters in image filenames.
\def\museincludegraphics{%
  \begingroup
  \catcode`\|=0
  \catcode`\\=12
  \catcode`\#=12
  \includegraphics[width=0.75\textwidth]
}
\author{Heyan Huang}
\date{\today}
\title{CS120 Lab 2}
\hypersetup{
  pdfkeywords={},
  pdfsubject={},
  pdfcreator={Emacs 24.3.50.1 (Org mode 8.2.5h)}}
\begin{document}

\maketitle
\tableofcontents


\section{CS120 Lab 2}
\label{sec-1}
\begin{frame}[fragile]\frametitle{Notes \& Lab1}
\label{sec-1-1}
\begin{itemize}
\item Appologize: 
\begin{itemize}
\item No \textbf{hard copy} after Assignment 1;
\item \textbf{cscheckin} version required;
\end{itemize}
\item Lab1 Distribution
\end{itemize}
\begin{center}
\begin{tabular}{lrrr}
\hline
S core & 8 & 9 & 10\\
\hline
Count (19) & 12 & 1 & 6\\
\hline
\end{tabular}
\end{center}
\begin{itemize}
\item \textbf{Comment} your code when it's necessary
\item \textbf{Indent}, Format the code file
\begin{itemize}
\item By default, for c++/c file, tab = 4 space
\end{itemize}
\item \textbf{Source code} example: \textbf{lab1.cpp}
\begin{itemize}
\item comment
\item indent
\item space format around operator
\end{itemize}
\item nano keyboard commands:
\begin{itemize}
\item \url{http://staffwww.fullcoll.edu/sedwards/Nano/UsefulNanoKeyCommands.html}
\item From this lab, I begin to use \textbf{Emacs}
\end{itemize}
\end{itemize}
\end{frame}
\begin{frame}[fragile]\frametitle{cmath library}
\label{sec-1-2}
\begin{itemize}
\item cmath library Functions
\url{http://www.cplusplus.com/reference/cmath/}
\begin{itemize}
\item pow(base, exponent): Returns base raised to the power exponent;
\item sqrt(x): Returns the square root of x;
\end{itemize}
\end{itemize}
\begin{center}
\begin{tabular}{llll}
\hline
Powre & Exponetnial & Trigonometric & Rounding\\
Functions & Functions & Functions & Remainder\\
\hline
pow & exp & cos & ceil\\
sqrt & log & sin & floor\\
 &  & tan & trunc\\
\hline
\end{tabular}
\end{center}
\begin{itemize}
\item try to remember the library names
\item try to remember the popular functions within each library
\end{itemize}
\end{frame}
% Emacs 24.3.50.1 (Org mode 8.2.5h)
\end{document}