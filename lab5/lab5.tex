% Created 2014-10-02 Thu 17:12
\documentclass[presentation]{beamer}
\usepackage[utf8]{inputenc}
\usepackage[T1]{fontenc}
\usepackage{fixltx2e}
\usepackage{graphicx}
\usepackage{longtable}
\usepackage{float}
\usepackage{wrapfig}
\usepackage{rotating}
\usepackage[normalem]{ulem}
\usepackage{amsmath}
\usepackage{textcomp}
\usepackage{marvosym}
\usepackage{wasysym}
\usepackage{amssymb}
\usepackage{hyperref}
\tolerance=1000
\usepackage{CJK}
\usepackage{indentfirst}
\usepackage[pdftex]{graphicx}
\usepackage{fancyhdr}
\usepackage[CJKbookmarks=true]{hyperref}
%% Define a museincludegraphics command, which is
%%   able to handle escaped special characters in image filenames.
\def\museincludegraphics{%
  \begingroup
  \catcode`\|=0
  \catcode`\\=12
  \catcode`\#=12
  \includegraphics[width=0.75\textwidth]
}
\usetheme{default}
\author{Heyan Huang}
\date{\today}
\title{CS120 Lab \alert{5} Section \alert{6}}
\hypersetup{
  pdfkeywords={},
  pdfsubject={},
  pdfcreator={Emacs 24.3.1 (Org mode 8.2.7c)}}
\begin{document}

\maketitle

\begin{frame}[fragile,label=sec-1]{Quiz for Week 4 \& 5 \alert{Answers}}
 \begin{itemize}
\item In C++ a library is a file containing code that can be used in other programs.
\end{itemize}
\\
\begin{itemize}
\item A conditional is used in a program to control whether or not a piece of code is executed.
\end{itemize}
\\
\begin{itemize}
\item The following code snippet will print It��s true.
\begin{verbatim}
int x = 15;
if(x + 20 > 2 * x){
    cout << "It��s true";
}
\end{verbatim}
\end{itemize}
\end{frame}

\begin{frame}[fragile,label=sec-2]{Quiz for Week 4 \& 5 \alert{Answers}: Type Conversion}
 \begin{itemize}
\item The expression 1 || 1, which represents true || true will evaluate to true.
\end{itemize}
\\
\begin{itemize}
\item The following code snippet will print the value 2.
\begin{verbatim}
int x = 25;
double y = 10;
cout << x/y;
\end{verbatim}
\end{itemize}
\end{frame}

\begin{frame}[fragile,label=sec-3]{Quiz for Week 4 \& 5 \alert{Answers}: Loop}
 \begin{itemize}
\item The following code snippet will print the word here exactly ten times.
\begin{verbatim}
int x = 0;
while (x < 10) {
    cout << "here";
}
\end{verbatim}
\end{itemize}
\\

\begin{itemize}
\item A while loop can always be rewritten as a do-while loop and vice versa.
\begin{itemize}
\item \alert{demonstrate} using codes
\end{itemize}
\end{itemize}
\end{frame}

\begin{frame}[fragile,label=sec-4]{Quiz for Week 4 \& 5 \alert{Answers}: Loop}
 \begin{itemize}
\item The following code snippet will print the word here exactly ten times.
\begin{verbatim}
int x = 0;
do {
    cout << "here";
} while (x < 10)
\end{verbatim}
\end{itemize}
\\

\begin{itemize}
\item After the following code snippet of code is done running x will have the value 9.
\begin{verbatim}
int x = 0;
while(x<10){
    cout << "here";
    x = x + 1;
}
\end{verbatim}
\end{itemize}
\end{frame}

\begin{frame}[fragile,label=sec-5]{Quiz for Week 4 \& 5 \alert{Answers}: Loop}
 \begin{itemize}
\item The following snippet of code will get stuck in an infinite loop.
\begin{verbatim}
int x = 1;
while(x != 10){
    cout << "here";
    x = x + 2;
}
\end{verbatim}
\end{itemize}
\end{frame}

\begin{frame}[label=sec-6]{Scores of Quiz3 \& Lab3 \& Lab4}
\\
\begin{itemize}
\item \alert{Lab 3} Distribution: 
\begin{itemize}
\item Section \alert{4}: 2 * \alert{9.5} / 21; 19 * \alert{10} / 21;
\item Section \alert{6}: 2 * \alert{9.5} / 19; 17 * \alert{10} / 19;
\end{itemize}
\end{itemize}
\\
\begin{itemize}
\item \alert{Quiz for Week 4} Distribution:
\end{itemize}
\begin{center}
\begin{tabular}{lrrrrrrr}
\hline
Score & 0 & 1 & 2 & 3 & 4 & 5 & Missed\\
\hline
Section \alert{4} Count (22) & 1 & 1 & 4 & 4 & 8 & 3 & 1\\
\hline
Section \alert{6} Count (24) & 0 & 1 & 2 & 7 & 5 & 5 & 4\\
\hline
\end{tabular}
\end{center}
\\
\begin{itemize}
\item \alert{Lab4} Distribution:
\end{itemize}
\begin{center}
\begin{tabular}{lrrrrrrrrrr}
\hline
Score & 6 & 6.5 & 7 & 7.5 & 8 & 10.5 & 11.5 & 12 & 12.5 & Missed\\
\hline
Section \alert{4} (22) & 1 & 1 &  & 2 &  & 0 & 9 & 4 & 1 & 4\\
\hline
Section \alert{6} (24) & 1 &  &  &  & 1 & 1 & 10 & 4 & 1 & 6\\
\hline
\end{tabular}
\end{center}
\end{frame}

\begin{frame}[label=sec-7]{Editor Command Set and Formatting}
\begin{itemize}
\item Nano: 
\begin{itemize}
\item search for \alert{nano cheat sheet}
\end{itemize}
\end{itemize}
\\
\begin{itemize}
\item Emacs: 
\begin{itemize}
\item enter emacs: \alert{emacs lab5.cpp}
\item \alert{indent} source program:
\begin{itemize}
\item \alert{C-x h} to select the whole buffer
\item hit \alert{Tab} key to autoindent the selected block of code
\item \alert{C-x C-c} to exit from emacs, and type "\alert{yes}" to \alert{save buffer}
\end{itemize}
\end{itemize}
\end{itemize}
\\
\begin{itemize}
\item Comment: 
\begin{itemize}
\item Block Comment is very important;
\item especially for this \alert{lab5}
\item \alert{Block comment} the parts who worked on which block/function
\end{itemize}
\end{itemize}
\end{frame}

\begin{frame}[label=sec-8]{\textless{}cmath\textgreater{} Library}
\begin{itemize}
\item \textless{}cmath\textgreater{} library Functions
\begin{itemize}
\item pow(base, exponent): Returns base raised to the power exponent;   
\begin{itemize}
\item prototype: \alert{double pow (double base, double exponent);}
\end{itemize}
\item sqrt(x): Returns the square root of x;
\begin{itemize}
\item prototype: \alert{double sqrt (double x);}
\end{itemize}
\end{itemize}
\end{itemize}
\begin{center}
\begin{tabular}{llll}
\hline
Powre & Exponetnial & Trigonometric & Rounding\\
Functions & Functions & Functions & Remainder\\
\hline
pow & exp & cos & ceil\\
sqrt & log & sin & floor\\
 &  & tan & trunc\\
\hline
\end{tabular}
\end{center}
\begin{itemize}
\item try to remember the \alert{Library names}
\end{itemize}
\\
\begin{itemize}
\item try to remember the most frequently used \alert{functions prototypes} within each library
\end{itemize}
\end{frame}

\begin{frame}[label=sec-9]{Lab 5 Specific Requirements}
\begin{itemize}
\item \alert{cscheckin}:
\begin{itemize}
\item \alert{Source Program} only
\end{itemize}
\end{itemize}
\\
\begin{itemize}
\item \alert{Hard Copy}:
\begin{itemize}
\item \alert{Source Program}
\item script \alert{Output} of the program
\end{itemize}
\end{itemize}
\\
\begin{itemize}
\item \alert{Exam Attention}:
\begin{itemize}
\item \alert{Exam} tomorrow \alert{Friday}, \alert{2014/10/3}
\end{itemize}
\end{itemize}
\end{frame}
% Emacs 24.3.1 (Org mode 8.2.7c)
\end{document}