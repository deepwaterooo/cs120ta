% Created 2014-10-16 Thu 12:23
\documentclass[presentation]{beamer}
\usepackage[utf8]{inputenc}
\usepackage[T1]{fontenc}
\usepackage{fixltx2e}
\usepackage{graphicx}
\usepackage{longtable}
\usepackage{float}
\usepackage{wrapfig}
\usepackage{rotating}
\usepackage[normalem]{ulem}
\usepackage{amsmath}
\usepackage{textcomp}
\usepackage{marvosym}
\usepackage{wasysym}
\usepackage{amssymb}
\usepackage{hyperref}
\tolerance=1000
\usepackage{CJK}
\usepackage{indentfirst}
\usepackage[pdftex]{graphicx}
\usepackage{fancyhdr}
\usepackage[CJKbookmarks=true]{hyperref}
%% Define a museincludegraphics command, which is
%%   able to handle escaped special characters in image filenames.
\def\museincludegraphics{%
  \begingroup
  \catcode`\|=0
  \catcode`\\=12
  \catcode`\#=12
  \includegraphics[width=0.75\textwidth]
}
\usetheme{default}
\author{Heyan Huang}
\date{\today}
\title{CS120 Lab \alert{7} Section \alert{6}}
\hypersetup{
  pdfkeywords={},
  pdfsubject={},
  pdfcreator={Emacs 24.3.1 (Org mode 8.2.7c)}}
\begin{document}

\maketitle

\begin{frame}[fragile,label=sec-1]{Quiz for Week 6 \alert{Answers}}
 All of the following questions are based on the class position, which is defined as:
\begin{verbatim}
class position {
private:
    int x;
    int y;
public:
    int distance();
};
\end{verbatim}
\begin{itemize}
\item Write the code to declare a single object named \alert{center} of type position:

\alert{position center;}
\item Write code to have the object named center call the \alert{distance()} function.

\alert{center.distance();}
\end{itemize}
\end{frame}

\begin{frame}[fragile,label=sec-2]{Quiz for Week 6 \alert{Answers} (continued)}
 All of the following questions are based on the class position, which is defined as:
\begin{verbatim}
class position {
private:
    int x;
    int y;
public:
    int distance();
};
\end{verbatim}
\begin{itemize}
\item What are the data members of the class?

\alert{x and y.}
\item What are the member functions (methods) of the class?

\alert{distance().}
\item If there was a constructor function for this class, what would the name of the function be?

\alert{position()}
\end{itemize}
\end{frame}

\begin{frame}[fragile,label=sec-3]{Quiz for Week 7 \alert{Answers}}
 (5 pt) Answer the following questions based on the following array declaration:
\begin{verbatim}
int hand[5];
\end{verbatim}
\begin{itemize}
\item How many elements (pieces of data) can the array hand store?
Answer: \alert{5}

\alert{Declare} an array: 

int \alert{arrayName} [ \alert{lengthOfArray} ]; 

and \alert{int} is arrayName's \alert{Data Type}.
\end{itemize}
\\
\begin{itemize}
\item The index of the \alert{first element} of the array?

Answer: \alert{0}

Array indices are \alert{0-based}.
\end{itemize}
\\

\begin{itemize}
\item What is the index of the \alert{last element} of the array?

Answer: \alert{4}

Since array indices are \alert{0-based}, index of the \alert{last} element of the array comes out to be \alert{lengthOfArray - 1}.
\end{itemize}

\\
\end{frame}

\begin{frame}[fragile,label=sec-4]{Quiz for Week 7 \alert{Answers} (continued)}
 (5 pt) Answer the following questions based on the following array declaration:
\begin{verbatim}
int hand[5];
\end{verbatim}
\begin{itemize}
\item Write a line of code to assign the second element of the array the value 7.

Answer: \alert{hand[ 1 ] = 7;}

Since array indices are \alert{0-based}, index of the \alert{second} element of the array comes out to be \alert{2 - 1}, and continue\ldots{}
\end{itemize}
\\
\begin{itemize}
\item Write a line of code to print the last element of the array.

Answer: \alert{cout << hand[ 4 ];}

Since array indices are \alert{0-based}, index of the \alert{last} element of the array comes out to be \alert{lengthOfArray - 1}, and continue\ldots{}
\end{itemize}
\\
\end{frame}

\begin{frame}[label=sec-5]{Scores of Quiz Week 6}
\\
\begin{itemize}
\item \alert{Lab 5} Score Distribution
\end{itemize}
\begin{center}
\begin{tabular}{lrrrrr}
\hline
Score & 8.5 & 9 & 9.5 & 10 & Missed\\
\hline
Section \alert{4} Count (22) & 1 & 2 & 8 & 9 & 2\\
\hline
Section \alert{6} Count (24) & 0 & 10 & 0 & 3 & 11\\
\hline
\end{tabular}
\end{center}
\\
\begin{itemize}
\item \alert{Quiz for Week 6} Distribution:
\end{itemize}
\begin{center}
\begin{tabular}{lrrrrrrr}
\hline
Score & 0 & 1 & 2 & 3 & 4 & 5 & Missed\\
\hline
Section \alert{4} Count (22) & 1 & 1 & 3 & 7 & 5 & 3 & 2\\
\hline
Section \alert{6} Count (24) & 0 & 0 & 2 & 9 & 3 & 3 & 7\\
\hline
\end{tabular}
\end{center}
\\
\begin{itemize}
\item \alert{Lab 6} Score Distribution
\end{itemize}
\begin{center}
\begin{tabular}{lrrrrr}
\hline
Score & 9 & 11 & 12 & 13 & Missed\\
\hline
Section \alert{4} Count (22) & 0 & 1 & 8 & 11 & 2\\
\hline
Section \alert{6} Count (24) & 2 & 2 & 1 & 9 & 10\\
\hline
\end{tabular}
\end{center}
\end{frame}

\begin{frame}[label=sec-6]{Lab 6 Randome Number Generator}
\begin{itemize}
\item \#include < \alert{cstdlib} >
\end{itemize}
\\
\begin{itemize}
\item Prototype: \alert{int rand (void);}
\item This number is generated by an algorithm that returns a sequence of apparently non-related numbers each time it is called.
\item Returns a pseudo-random integral number in the range between 0 and RAND\textunderscore MAX.
\item RAND\textunderscore MAX is a constant defined in <cstdlib>.
\end{itemize}
\\
\begin{itemize}
\item Examples:
\begin{itemize}
\item v1 = rand() \% 100;
\item v2 = rand() \% 100 + 1;
\item v3 = rand() \% 30 + 1985;
\end{itemize}
\end{itemize}
\\ 
\begin{itemize}
\item Seed
\begin{itemize}
\item This algorithm uses a seed to generate the series, which should be initialized to some distinctive value using function srand.
\item \emph{* initialize random seed: *}

\alert{srand (time(NULL));}
\end{itemize}
\end{itemize}
\end{frame}
\begin{frame}[label=sec-7]{Lab 6 Specific Requirements}
\begin{itemize}
\item \alert{cscheckin}:
\begin{itemize}
\item \alert{Source Program} only
\item program name: \alert{Lab7Sec6.cpp}
\end{itemize}
\end{itemize}
\\
\begin{itemize}
\item \alert{Hard Copy}:
\begin{itemize}
\item \alert{Source Program}: \alert{Lab7Sec6.cpp}
\item \alert{Script Output} of the program:
\end{itemize}
\end{itemize}
\\
\begin{itemize}
\item \alert{Extra Credits}:
\begin{itemize}
\item \alert{Extra Work} are required in order to extra points.
\item It won't be easy to get all \alert{3} extra credits, so make sure you make effort to earn them.
\end{itemize}
\end{itemize}
\end{frame}
% Emacs 24.3.1 (Org mode 8.2.7c)
\end{document}