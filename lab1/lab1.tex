% Created 2014-09-22 Mon 20:44
\documentclass[presentation]{beamer}
\usepackage[utf8]{inputenc}
\usepackage[T1]{fontenc}
\usepackage{fixltx2e}
\usepackage{graphicx}
\usepackage{longtable}
\usepackage{float}
\usepackage{wrapfig}
\usepackage{rotating}
\usepackage[normalem]{ulem}
\usepackage{amsmath}
\usepackage{textcomp}
\usepackage{marvosym}
\usepackage{wasysym}
\usepackage{amssymb}
\usepackage{hyperref}
\tolerance=1000
\usepackage{CJK}
\usepackage{indentfirst}
\usepackage[pdftex]{graphicx}
\usepackage{fancyhdr}
\usepackage[CJKbookmarks=true]{hyperref}
%% Define a museincludegraphics command, which is
%%   able to handle escaped special characters in image filenames.
\def\museincludegraphics{%
  \begingroup
  \catcode`\|=0
  \catcode`\\=12
  \catcode`\#=12
  \includegraphics[width=0.75\textwidth]
}
\usetheme{default}
\author{Heyan Huang}
\date{\today}
\title{CS120 Lab 1}
\hypersetup{
  pdfkeywords={},
  pdfsubject={},
  pdfcreator={Emacs 24.3.1 (Org mode 8.2.7c)}}
\begin{document}

\maketitle

\begin{frame}[label=sec-1]{Introduce Myself}
\begin{itemize}
\item Name: \alert{Heyan Huang}
\item Major: Statistics \& \alert{Computer Science}
\item Experience: OPT Statistical industry experience, \& 2 years cs course study, I use Emacs editor during lab sections;
\item Contact Email: \alert{heyanh@vandals.uidaho.edu} \\ Prefer: \alert{heyan.huang2010@gmail.com}
\item Technical Help: \alert{211 CSAC}, mainly \alert{MTWF} hours, try to find me there\textasciitilde{}~!!
\end{itemize}
\end{frame}

\begin{frame}[label=sec-2]{Lab Rules}
\begin{block}{\alert{Quiz}:}
\begin{itemize}
\item At the beginning of the lab, you will have short answer \alert{quizs};
\item Count up to 15\% for final score, as well as indicating attendence;
\end{itemize}
\end{block}
\begin{block}{\alert{No Late}:}
\begin{itemize}
\item Labs are due at the \alert{end of lab section};
\item Homeworks are due on \alert{followed Monday};
\end{itemize}
\end{block}
\begin{block}{\alert{Turn-in Requirements}:}
\begin{itemize}
\item for \alert{Lab} and for \alert{Homeworks};
\item \alert{electronic versions} is required;
\item \alert{hard copy} as an extra;
\item \alert{Electronic} copy can be checked in through "\alert{cscheckin}" command. Specify \alert{filename} and \alert{course folder}, eg. "\alert{cs120}"
\end{itemize}
\end{block}
\begin{block}{\alert{No Cheating}:}
\begin{itemize}
\item No \alert{copying} codes or program Results from students;
\item No \alert{copying} codes or program Results from website without understanding;
\item \alert{hard copy} results must \alert{match} your \alert{Electronic} program's results;
\end{itemize}
\end{block}
\end{frame}

\begin{frame}[label=sec-3]{Lab Rules}
\begin{itemize}
\item No talking to neighbors; Either raise your questions or answer mine;
\item I promote \alert{independent \& inspiring thinking}. You are free to ask questions with consideration for classmates.
\item If you are doing lab sections well, I will be more than happy to teach you some other skills, like introducing \alert{Emacs editor}, including some practical \alert{interview questions} and coding practise in lab section.
\item If you have difficulty with lab or homework, I am willing to help in CSAC, and I would feel happy to see you make progress.
\end{itemize}
\end{frame}

\begin{frame}[label=sec-4]{Linux Commands}
\begin{block}{\alert{pwd}:  \alert{p} rint \alert{w} orking \alert{d} irectory}
\end{block}
\begin{block}{\alert{ls}:  \alert{l} i \alert{s} t files and Directories}
\end{block}
\begin{block}{\alert{mkdir}:  \alert{m} a \alert{k} e \alert{dir} ectory}
\end{block}
\begin{block}{\alert{cd}:  \alert{c} hange your working \alert{d} irectory}
\begin{itemize}
\item The "\alert{.}" symbol refers to the working directory;
\item The "\alert{..}" symbol refers to the working directory's parent directory;
\item The "\alert{./}" symbol means execute script from my current directory. Dot (\alert{.}), or current directory is never on the \alert{PATH} ( \alert{echo \$PATH} to check this ) for security reasons and it never should be.
\end{itemize}
\end{block}
\begin{block}{\alert{script}: make typescript of terminal session}
\end{block}
\begin{block}{\alert{exit}: The exit operation typically performs clean-up operations within the process space before returning control back to the operating system. (source: Wikipedia)}
\end{block}
\end{frame}

\begin{frame}[label=sec-5]{Programming Environment Introcution}
\begin{enumerate}
\item Use \alert{putty} to log into \alert{wormulon.cs.uidaho.edu}. Use your \alert{Vandal Username} and \alert{password} to log on.
\item Use the \alert{mkdir} command to create a directory called \alert{labs}.
\item Use the \alert{cd} command to move into the new \alert{labs} directory.
\item Use emacs/nano to create a file called \alert{fortune.cpp} and write the Fortune Teller program in the file. Add a block of comments to the beginning of the program that lists your name, section number, date, and the assignment number.
\item Use \alert{g++} to compile the Fortune Teller program. You may need to type \alert{./a.out} to run the program.
\end{enumerate}
\end{frame}

\begin{frame}[label=sec-6]{Sample Program}
\begin{enumerate}
\item Modifications: Print a \alert{welcome message} at the beginning of the program; Change the program so that the fortunes are \alert{different}. \alert{Make up your own fortunes}, try to keep them interesting.
\item Script: Use the \alert{script} (make typescript of terminal session) command to create a printable output file. The command \alert{script lab1output} will create a file called \alert{lab1output}.
\item View Results: use the commands \alert{pwd} and \alert{ls} to show the current directory and its contents. Finally, use the \alert{exit} command to end the script.
\item Final: Now You have a file called \alert{fortune.cpp} containing the Fortune Teller program and a file called \alert{lab1output} containing a 'transcript' of you running the Fortune Teller program and the pwd and ls commands. Print both files (using the \alert{lpr} command) and turn them in.
\end{enumerate}
\end{frame}
% Emacs 24.3.1 (Org mode 8.2.7c)
\end{document}